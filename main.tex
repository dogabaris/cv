\documentclass[margin,line]{resume}
 
\usepackage[utf8]{inputenc}

\usepackage[T1]{fontenc}
\usepackage{fontawesome}
\usepackage{graphicx,wrapfig}
\usepackage{url}
\usepackage[colorlinks=true, pdfstartview=FitV, linkcolor=blue, citecolor=blue, urlcolor=blue]{hyperref}
\pdfcompresslevel=9


\begin{document}{\sc \Large Doğa Barış Özdemir}
\begin{resume}

% === PICTURE ===

    \vspace{0.5cm}
    \begin{wrapfigure}{R}{0.15\textwidth}
         \vspace{-0.9cm}
        \begin{center}
        \includegraphics[width=0.15\textwidth]{pp.jpg}
        \end{center}
         \vspace{-1cm}
    \end{wrapfigure}

% === PERSONAL INFO ===
 
    \section{\mysidestyle Personal\\Information}
    Doğa Barış Özdemir, Computer Engineer \\
    İstanbul, Turkey\\ 
    (+90) 543 341 341 5 \\
    \href{https://www.linkedin.com/in/dogabarisozdemir/}{LinkedIn} \\
    \href{https://github.com/dogabaris}{Github} \\ 
    \href{https://play.google.com/store/apps/developer?id=bigApps}{Google Play Store} 
    
   
% === OBJECTIVE ===

    \section{\mysidestyle Professional Objective}
    Passionate software developer with 4+ years experience who likes to do research on solving problems. Highly flexible team player who likes brainstorming with teammates. Solution-oriented. Tech agnostic. Willing to integrate academical works into enjoyable software products and services. Open to learning. Developer who sets goals to be successful.
 
% === SKILLS ===
       
    \section{\mysidestyle Skills}\vspace{2mm}       
    \begin{description}
   				\item[Software frameworks:] .Net, Cobra (golang), React, Angular, NodeJs, Android, Elasticsearch.
				\item[Programming languages:] C\#, Golang, Javascript, Java.
				\item[DevOps and containerization:] Git, Teamcity, Jenkins, Kubernetes, Docker.
				\item[Message brokers:] RabbitMQ.
				\item[Database query languages:] NoSql (MongoDb), Sql, GraphDb (Neo4j).
				\item[Work methodologies:] Scrum, Agile, Kanban.
				\item[Operating systems:] macOS, Windows, Linux.
				\item[Scientific softwares] Tensorflow, Keras.
				\item[Languages:] Turkish, English.
    \end{description}
     
% === Education ===
    
       \section{\mysidestyle Education} 
 	   Master's Degree with Thesis in Artificial Neural Networks on Graph Databases, Computer Engineering, 2020,   Çanakkale Onsekiz Mart University (Turkey)\\
       Bachelor's Degree, Computer Engineering, 2017, Kocaeli University (Turkey)\\

% === Publications ===

      \section{\mysidestyle Publications}\vspace{4mm}  
      \begin{list2}
        \item \textbf{The Use of Graph Databases for Artificial Neural Networks}  \href{https://dergipark.org.tr/tr/pub/jarnas/issue/60593/890552}{[link]}, Çanakkale Onsekiz Mart University Journal of Graduate School of Natural and Applied Sciences, 2021 \\
        \vspace{1mm}
        \item \textbf{Design of an Arduino Based Low-Cost Air Conditioning Automation Device Using Temperature Humidity Index (THI)} \href{https://dergipark.org.tr/tr/pub/turkjans/issue/56111/694653}{[link]}, Türk Tarım ve Doğa Bilimleri Dergisi, 2019 \\
        \vspace{1mm}
        \item \textbf{Bir E-burun Sisteminin Arduino Tabanlı Dönüşümünün Yapılması (Arduino Based Transformation of an E-nose System)}, 1st International Congress on Agricultural Structures and Irrigation, 2018 \\
 		\vspace{1mm}
        \item \textbf{Görüntülerdeki Araba Nesnelerinin Belirlenmesi İçin Derin
Öğrenme ile Bir Model Eğitilmesi (Training a Model with Deep Learning to Identify Car Objects in Images)} \href{https://ab.org.tr/kitap/ab18.pdf}{[link]}, Akademik Bilişim, 2018 \\

      \end{list2}

 	   	\vfill
 	   
% === HISTORY ===

 	    \section{\mysidestyle Employment History}\vspace{2mm}   	    	          
       \begin{description}       
   		    \item[Software Developer]\small{Hepsiburada \hfill \textsl{February 2020-Present}}\\
       		Worked as a backend software  developer in the Order Management System team of Hepsiburada, one of the leading platforms of e-commerce in Turkey. Took part in the design and development of fulfillment and claim systems for all orders. Took part in the development of systems based on Domain Driven Design with CQRS pattern in micro service architecture. Used C\#, Golang, React, Javascript, Nodejs, Python, Message Broker technologies. Been involved in the development of pages used by millions of customers, administrators and vendors. Made improvements in micro frontend architecture. Worked with NoSql, Sql databases to store data. Used ElasticSearch search engine.
       		
 	      	\vspace{2mm}
    		\item[Software Developer]\small{Shopi \hfill \textsl{March 2019-February 2020}}\\
    		Developed integration system architecture and apis. Fixed backend issues and developed solutions to support new features with customer datas. Integrated Verifone and Ingenico PoS (Point of Sale) payment systems via middlewares to Shopi products. Worked with various size retail companies such as Dolce\&Gabbana to Tefal. Designed solutions with developer teams to satisfy customers’ requests and also made space for new feature developments. Set goals and managed them with Scrum/Agile methodologies.
 			\vspace{2mm}
 				
 			\item[Fullstack Software Developer]\small{Argelog \hfill \textsl{November 2017-January 2019}}\\
 				    Developed a live chat system for R\&D projects management on SignalR/Socket backed with .Net Backend. Developed frontend with Angular4+ and AngularJs. Researched and developed Windows application and mobile versions of the project. Maintained stability with unit tests and load tests. Used Nhibernate as framework. Remote worked with agile/scrum methodologies on projects.
 			\vspace{2mm}
 				
     		\item[Intern Software Developer]\small{Intertech \hfill \textsl{August 2016-September 2016}}\\
   			Examined and programmed case studies of various data conversions used in the infrastructures of the banks Intertech serve. 	      
 			\vspace{2mm}
            
            \item[Intern Software Developer]\small{Kocaeli University Embedded Systems Lab \hfill \textsl{July 2015-September 2015}}\\
    			Took part in the project of designing and developing an android and python based notice board system for faculty members' rooms.

			\vspace{2mm}
    				
    \end{description} 
\end{resume}   
\end{document}